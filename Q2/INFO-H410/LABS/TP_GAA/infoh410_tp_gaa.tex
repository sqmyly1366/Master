\documentclass[11pt,a4paper]{article}
\usepackage[utf8]{inputenc}
\usepackage[T1]{fontenc}
\usepackage{amsthm} %numéroter les questions
\usepackage[english]{babel}
\usepackage{datetime}
\usepackage{xspace} % typographie IN
\usepackage{hyperref}% hyperliens
\usepackage[all]{hypcap} %lien pointe en haut des figures

\usepackage{subcaption}
\usepackage{fancyhdr}% en têtes
\usepackage{graphicx} %include pictures
\usepackage{pgfplots}
\usepackage[]{algorithm2e}

\usepackage{tikz}
\usetikzlibrary{calc}
\usetikzlibrary{babel}
\usepackage{circuitikz}
% \usepackage{gnuplottex}
\usepackage{float}
\usepackage{ifthen}

\usepackage[top=1.3 in, bottom=1.3 in, left=1.3 in, right=1.3 in]{geometry}
\usepackage[]{pdfpages}
\usepackage[]{attachfile}

\usepackage{amsmath}
\usepackage{enumitem}
\setlist[enumerate]{label=\alph*)}% If you want only the x-th level to use this format, use '[enumerate,x]'
\usepackage{multirow}

\usepackage{aeguill} %guillemets


%%%%%%%%%%%%%%%%%%%%%%%%%%%%%%%%%%%%%%%%%%%%%%%%%%%%%%%%%%%%%%%%%%%%%%%%%%%%%%%%%%%%%%%%%%%
% READ THIS BEFORE CHANGING ANYTHING
%
%
% - TP number: can be changed using the command
%		\def \tpnumber {TP 1 }
%
% - Version: controlled by a new command:
%		\newcommand{\version}{v1.0.0}
%
% - Booleans: there are three booleans used in this document:
%	- 'corrige', controlled by defining the variable 'corrige'
% You can define those variables in a makefile using such a command:
% pdflatex -shell-escape -jobname="infoh410_tp1" "\def\corrige{} \input{infoh410_tp1.tex}"
%
%%%%%%%%%%%%%%%%%%%%%%%%%%%%%%%%%%%%%%%%%%%%%%%%%%%%%%%%%%%%%%%%%%%%%%%%%%%%%%%%%%%%%%%%%%%

%Numero du TP :
\def \tpnumber {TP GAA (Genetic Algorithms and Ant System) }

\newcommand{\version}{v1.0.0}


% ########     #####      #####    ##         #########     ###     ##      ##  
% ##     ##  ##     ##  ##     ##  ##         ##           ## ##    ###     ##  
% ##     ##  ##     ##  ##     ##  ##         ##          ##   ##   ## ##   ##  
% ########   ##     ##  ##     ##  ##         ######     ##     ##  ##  ##  ##  
% ##     ##  ##     ##  ##     ##  ##         ##         #########  ##   ## ##  
% ##     ##  ##     ##  ##     ##  ##         ##         ##     ##  ##     ###  
% ########     #####      #####    #########  #########  ##     ##  ##      ##  
\newboolean{corrige}
\ifx\correction\undefined
\setboolean{corrige}{false}% pas de corrigé
\else
\setboolean{corrige}{true}%corrigé
\fi

% \setboolean{corrige}{true}


\definecolor{darkblue}{rgb}{0,0,0.5}

\pdfinfo{
/Author (ULB -- CoDe/IRIDIA)
/Title (\tpnumber, INFO-H-410)
/ModDate (D:\pdfdate)
}

\hypersetup{
pdftitle={\tpnumber [INFO-H-410] Techniques of AI},
pdfauthor={ULB -- CoDe/IRIDIA},
pdfsubject={}
}

\theoremstyle{definition}% questions pas en italique
\newtheorem{Q}{Question}[] % numéroter les questions [section] ou non []


%  ######     #####    ##       ##  ##       ##     ###     ##      ##  ######      #######   
% ##     ##  ##     ##  ###     ###  ###     ###    ## ##    ###     ##  ##    ##   ##     ##  
% ##         ##     ##  ## ## ## ##  ## ## ## ##   ##   ##   ## ##   ##  ##     ##  ##         
% ##         ##     ##  ##  ###  ##  ##  ###  ##  ##     ##  ##  ##  ##  ##     ##   #######   
% ##         ##     ##  ##       ##  ##       ##  #########  ##   ## ##  ##     ##         ##  
% ##     ##  ##     ##  ##       ##  ##       ##  ##     ##  ##     ###  ##    ##   ##     ##  
%  #######     #####    ##       ##  ##       ##  ##     ##  ##      ##  ######      #######   
\newcommand{\reponse}[1]{% pour intégrer une réponse : \reponse{texte} : sera inclus si \boolean{corrige}
\ifthenelse {\boolean{corrige}} {\paragraph{Answer:} \color{darkblue}   #1\color{black}} {}
}

\newcommand{\addcontentslinenono}[4]{\addtocontents{#1}{\protect\contentsline{#2}{#3}{#4}{}}}


%  #######   ##########  ##    ##   ##         #########  
% ##     ##      ##       ##  ##    ##         ##         
% ##             ##        ####     ##         ##         
%  #######       ##         ##      ##         ######     
%        ##      ##         ##      ##         ##         
% ##     ##      ##         ##      ##         ##         
%  #######       ##         ##      #########  #########  
%% fancy header & foot
\pagestyle{fancy}
\lhead{[INFO-H-410] Techniques of AI\\ \tpnumber}
\rhead{\version\\ page \thepage}
\chead{\ifthenelse{\boolean{corrige}}{Correction}{}}
%%

\setlength{\parskip}{0.2cm plus2mm minus1mm} %espacement entre §
\setlength{\parindent}{0pt}

% ##########  ########   ##########  ##         #########  
%     ##         ##          ##      ##         ##         
%     ##         ##          ##      ##         ##         
%     ##         ##          ##      ##         ######     
%     ##         ##          ##      ##         ##         
%     ##         ##          ##      ##         ##         
%     ##      ########       ##      #########  ######### 
\date{\vspace{-1.7cm}\version}
\title{\vspace{-2cm} \tpnumber \\ Techniques of AI [INFO-H-410] \ifthenelse{\boolean{corrige}}{~\\Correction}{}}


\begin{document}
\selectlanguage{english}

\maketitle

\fbox{
    \parbox{\textwidth}{
    \label{source}
\footnotesize{
Source files, code templates and corrections related to practical sessions can be found on the UV 
or on github (\url{https://github.com/iridia-ulb/INFOH410}).
}}}

\paragraph{Genetic Algorithms}

\begin{Q}
\begin{enumerate}
\item Based on the two candidate solutions 11101001000 and 
00001010101, apply the following crossover procedures:
\begin{itemize}
	\item Single-point crossover: 11111000000
	\item Two-point crossover: 	  00111110000
	\item Uniform crossover: 	  10011010011
\end{itemize}
\item Implement a genetic algorithm that starts with an initial
population of randomly generated strings, and that returns a string 
as close as possible to the target string ``Hello, World". 
You can use the provided template in which the \texttt{fitness}, 
\texttt{mutate},  and \texttt{crossover} functions are missing. Your goal is to 
implement these functions, together with the core logic of the 
algorithm.  
\end{enumerate}

\reponse{
    (a) 11101001000 and 00001010101, give, spc: 00001001000, tpc: 11001011000, and uc: 01101001001 

    (b) see github
}

\end{Q}

\paragraph{Ant system}

\begin{Q}
\begin{enumerate}
% \item What kind of algorithm is Ant system (AS)? Does AS guarantees 
%to eventually find the optimal solution or to determine that no 
%solution exists?
 \item Is it possible to use AS with $\{\alpha=0, \beta=1\}$ or  $\{\alpha=1, \beta=0\}$? 
     What is the effect in both cases?
 \item How does $\rho$ relates with the exploration and exploitation performed by AS? What 
     about $\alpha$ and $\beta$?
% \item In class we used a minimization problem (TSP) as example, 
%what modifications are needed to solve a maximization problem with 
%AS?
\end{enumerate}

\reponse{
    (a) Both options are possible, but maybe not a great idea. If AS is set with parameters 
    $\alpha = 0$ and $\beta = 1$, pheromones will be neglected from the transition rule that 
    ants employ to construct solutions, meaning that they will not be learning from high 
    quality solutions found in previous iterations. On the other hand, if AS is set with 
    $\alpha = 1$ and $\beta = 0$, ants solutions construction will be biased only by pheromone,
    which may end up in low quality solutions, since the heuristic information also allows 
    to select high quality solution components, but from a greedy perspective. In a nutshell, 
    AS performance is generally better when the algorithm can profit both from the numerical 
    information learned in past iterations (pheromones) and from problem-dependent information 
    (heuristic information).


    (b) $\rho$ controls the speed at which pheromone decreases. For large $\rho$ values, the 
    value of pheromones will decrease faster resulting in a more exploratory behavior, and for 
    small $\rho$ values, the value of pheromones will decrease slower leading to a more 
    exploitative behavior. In the case of $\alpha$ and $\beta$, these two parameters control 
    the relative influence of the pheromones and the heuristic information. When both values 
    are large, the algorithm will behave more exploitative because ants will be biased towards 
    solutions that have either high pheromone values or better heuristic information. 
    Conversely, when both $\alpha$ and $\beta$ are small, the probability associated to all 
    feasible solution components will be more even, resulting in a more exploratory behavior.
}

\end{Q}
\begin{Q}
 Assume the following symmetric Traveling Salesman Problem (TSP)
 instance:
   \begin{figure}[ht]
   \begin{subfigure}{0.31\textwidth}
   \begin{tabular} {| l | c |c | c | c | c |}
   \hline
        &  A  &  B  &  C  &  D  & E   \\
    \hline
      A & $-$ &  1  &  2  &  2  & 6   \\
      B &  1  & $-$ &  6  &  8  & 10  \\  
      C &  2  &  6  & $-$ &  12 & 4   \\
      D &  2  &  8  & 12  & $-$ & 1   \\
      E &  6  & 10  &  4  &   1 & $-$  \\
    \hline
   \end{tabular}
   \caption{TSP instance (tsp)}
   \end{subfigure}
   ~~~~~~~~
   \begin{subfigure}{0.45\textwidth}
   \centering
   \begin{tabular} {| l | c |c | c | c | c |}
   \hline
        &  A  &  B  &  C  &  D  & E   \\
    \hline
      A & $-$    &  0.56  &  0.66  &  0.60  & 0.50   \\
      B &  0.56  & $-$    &  0.60  &  0.56  & 0.60  \\  
      C &  0.66  &  0.60  & $-$    &  0.50  & 0.56   \\
      D &  0.60  &  0.56  &  0.50  & $-$    & 0.66   \\
      E &  0.50  &  0.60  &  0.56  &  0.66  & $-$  \\
    \hline
   \end{tabular}
   \caption{Pheromone matrix ($\tau$) (iteration 1) }
   \end{subfigure}
   \end{figure}
   
   An Ant System algorithm is applied to this instance using $\alpha=2$, $\beta=1$, $\rho=0.5$, $\#ants=3$, $\eta_{ij}=1/tsp_{ij}$ and $\tau_0=1$, 
   after the first iteration the pheromone matrix ($\tau$) is the one given in the figure above. 
 \begin{enumerate}
     \item What is the meaning of the values in $\tau$? Why $\tau_{C,D}=0.5$?
     \item Use this information to generate the first solution of 
     iteration 2 starting from city D. Use random numbers: $\{0.80, 0.27, 0.88, 0.47, 0.05, 0.98, 0.23, 0.06 \}$
     
 \item Actually, the following solutions generated by the algorithm are $AEDCBA$ with $cost=26$ and $DECBAD$ with $cost=14$. Update the pheromone using this information (disregard the solution of the previous question).
     \item After 12 iterations the pheromone matrix is the following:
        \begin{figure}[ht]
           \centering
           \begin{tabular} {| l | c |c | c | c | c |}
            \hline
                &  A  &  B  &  C  &  D  & E   \\
            \hline
              A & $-$     &  0.4285  &  0.0004  &  0.4285  & 0.0003   \\
              B &  0.4285 & $-$      &  0.4286  &  0.0003  & 0.0003  \\  
              C &  0.0004 &  0.4286  & $-$      &  0.0003  & 0.4285   \\
              D &  0.4285 &  0.0003  &  0.0003  & $-$      & 0.4286   \\
              E &  0.0003 &  0.0003  &  0.4285  &  0.4286  & $-$  \\
            \hline
            \end{tabular}
            \caption{Pheromone matrix ($\tau$) (iteration 12)}
        \end{figure}
         
         Would you advice to continue executing more iterations? Why?

 \textit{Remember: the tour length is computed starting and ending in the same city.}
 %\item How would you apply AS to the Graph Coloring Problem(GCP)? What would be the decision variables? How to define the pheromone? Is it possible to define heuristic information?\\
 
\item Using the provided template, implement the \verb!compute_probability_matrix!, \verb!evaporate_pheromone!, \verb!deposit_pheromone! and \verb!get_next_city! functions of ACO for solving the TSP.

  %\textit{Note: The objective of the GCP is to find the minimum number of different colors with which is possible to assign a color to each node of a given graph without having adjacent nodes with the same color.} 
 \end{enumerate}

 \reponse{
    (a) $\tau_{i,j}$ indicates the desirability of adding edge $i$ to the tour being constructed 
    by an ant when in edge $j$. Since $\rho=0.5$ and $\tau_0=1$, $\tau_{C,D}=0.5$ indicates that 
    the edge $(C,D)$ was not part of the tour constructed by any ant.

    (b) 
    Using the random proportional rule of AS, 
    $$p^k_{i,j}(t)= \frac{\tau_{i,j}(t)^{\alpha}\cdot \eta_{i,j}^{\beta}}{\sum_{l\in N^k_i}  \tau_{i,l}(t)^{\alpha}\cdot \eta_{i,l}^{\beta} },$$
    we compute the probability of adding each feasible city, and using the roulette wheel
    mechanism and the set of random numbers previously generated, we select one of the city to be added to the tour.
    
    This is done as follows: \\
     	Initial city: D

		Next city:\\
	     \begin{tabular} { l | c |c | c | c | c |}
			\cline{2-6}
			& $j=$A  &  $j=$B  &  $j=$C  &  $j=$D  &  $j=$E\\
			\hline
			probabilities ($p^{k=1}_{D,j}(t=2)$) &  0.27 & 0.06 & 0.031 & 0 & 0.64\\
			\hline
            roulette wheel ($rnd= 0.80$) & 0.27 & 0.32 & 0.36 & 0 & \textbf{1}\\
			\hline
		\end{tabular}\\

        Next city:\\
	     \begin{tabular} { l | c |c | c | c | c |}
			\cline{2-6}
			& $j=$A  &  $j=$B  &  $j=$C  &  $j=$D  &  $j=$E\\
			\hline
			probabilities ($p^{k=1}_{E,j}(t=2)$) &  0.266  & 0.23  & 0.50  &  0  & 0\\
			\hline
            roulette wheel ($rnd= 0.27$) & 0.266  & \textbf{0.49}  & 1 & 0 & 0\\
			\hline
		\end{tabular}\\

		Next city:\\
	     \begin{tabular} { l | c |c | c | c | c |}
			\cline{2-6}
			& $j=$A  &  $j=$B  &  $j=$C  &  $j=$D  &  $j=$E\\
			\hline
			probabilities ($p^{k=1}_{A,j}(t=2)$) &  0.84 & 0 & 0.16  &  0  & 0\\
			\hline
            roulette wheel ($rnd= 0.88$) & 0.84 & 0 & \textbf{1}  &  0  & 0\\
			\hline
		\end{tabular}

        The first solution constructed at iteration $t=2$ is: \textit{DEBCAD} %\textit{DEACBD}


    (c) Using the pheromone update rule of AS, 
    $$\tau_{i,j}(t)= (1-\rho)\cdot\tau_{i,j}(t-1)+ \sum_{k=1}^{m} \Delta\tau_{i,j}^k,$$ we get:

	 $(AB) = (0.5)\cdot0.56 + (1/26)+(1/14) = 0.38989011$\\
	 $(AC) = (0.5)\cdot0.66 = 0.33$\\
	 $(AD) = (0.5)\cdot0.60 + (1/14) = 0.371428571$\\
	 $(AE) = (0.5)\cdot0.50 + (1/26) = 0.288461538$\\
	 $(BC) = (0.5)\cdot0.60 + (1/26)+(1/14) = 0.4098901$\\
	 $(BD) = (0.5)\cdot0.56 = 0.28$\\
	 $(BE) = (0.5)\cdot0.60 = 0.3 $\\
	 $(CD) = (0.5)\cdot0.50 + (1/26) = 0.288461538$\\
	 $(CE) = (0.5)\cdot0.56 + (1/14) = 0.351428571$\\
	 $(DE) = (0.5)\cdot0.66 + (1/26)+(1/14) = 0.43989011$\\
     
    (d)
	At this point the algorithm is stagnated, which means that the probability of 
    constructing a set of solutions different from the one constructed in the last 
    iteration is quite low. In this example, the algorithm will be constructing solutions
     	$$
     	\begin{aligned}
     		&ABCEDA, \\
     		&ADECBA, \\
     		&BADECB, \\
     		&BDECAB, \\
     		&CBADEC, \\
     		&CEDABC, \\
     		&DABCED, \\
     		&DECBAD, \\
     		&ECBADE, \;\mathrm{and}\\
     		&EDABCE
     	\end{aligned}
     	$$
     	over and over again.

        (e) see github for implementation
 }

 \end{Q}

\clearpage
\paragraph{Reminder}

\begin{algorithm}[h!]
	\KwData{$Pop_{size}$, $Num_{gen}$, 
		$Crossover_{proc}$, $Mutation_{proc}$}
		~\\
		\For{i $\in$ ${1, .., Pop_{size}}$ }{
			Population$\left[i\right]$ $\leftarrow$ RandomSolution()}
			Evaluate(Population)\\
			$S_{best}$ $\leftarrow$ GetBestSolution(Population)\\
			\For{gen $\in$ ${1, .., Num_{gen}}$ }{
				Parents $\leftarrow$ Selection(Population)\\
				Children $\leftarrow$ $\phi$ \\
				\For{$Parent_1$, $Parent_2$ $\in$ Parents}{
					$Children_1$, $Children_2$ $\leftarrow$ 
					Crossover($Parent_1$, $Parent_2$, 
					$Crossover_{proc}$)\\
					$Children_1$ $\leftarrow$ Mutate($Children_1$, 
					$Mutation_{proc}$) \\
					$Children_2$ $\leftarrow$ Mutate($Children_2$, 
					$Mutation_{proc}$)
					}
			Evaluate(Children)\\
			$P_{best}$ $\leftarrow$ 
			GetBestSolution(Children)\\
			\If{$P_{best}$ $>$ $S_{best}$}{$S_{best}$ 
				$\leftarrow$ $P_{best}$}
			Population $\leftarrow$ Children\\
			}
	\caption{Pseudocode for a simple \textbf{\large{genetic 
	algorithm}}. 
	$Pop_{size}$ 
	is the number of individuals in the population, 
	$Num_{gen}$ is the number of generations, 
		$Crossover_{proc}$ is a crossover procedure, and 
		$Mutation_{proc}$ is a mutation procedure.}
	\label{alg:ga}
\end{algorithm}
						
						
						
\begin{algorithm}[h!]
	\KwData{$n$, $m$, $\alpha$, $\beta$, $\rho$, $\tau_0$}
	~\\
	\While{!termination()}{
		\For{k $\in$ ${1, .., m}$}{
			ants$\left[k\right]\left[1\right]$ $\leftarrow$ 
			SelectRandomCity$()$\\
			\For{i $\in$ ${2, n}$}{
				ants$\left[k\right]\left[i\right]$ $\leftarrow$ 
				ASDecisionRule$($ants, i$)$	
			}
			ants$\left[k\right]\left[n+1\right]$ $\leftarrow$ 
			ants$\left[k\right]\left[1\right]$
		}
		UpdatePheromones$($ants$)$
	}
	\caption{Pseudocode for the \textbf{\large{Ant System 
	algorithm}}. $n$ is 
	the size 
	of the problem, $m$ the number of ants, $\alpha$ and $\beta$ are 
	parameters of the solution construction procedure ASDecisionRule 
	(see Eq.~\ref{eq:deci}), 
	$\rho$ is the parameter of the pheromone update procedure 
	UpdatePheromones (see Eq.~\ref{eq:phero}), and $\tau_0$ is the 
	initial value of the 
	pheromones}
	\label{alg:as}
\end{algorithm}

\begin{equation}\label{eq:deci}
	ASDecisionRule \rightarrow p_{ij}^k(t) = 
	\frac{\left[\tau_{ij}\right]^{\alpha}\cdot\left[\eta_{ij}\right]^\beta}
	 {\sum_{l \in N_i^k} 
	\left[\tau_{il}\right]^{\alpha}\cdot\left[\eta_{il}\right]^\beta},
	\qquad \text{if} \quad j\in N_i^k
\end{equation}

\begin{equation}\label{eq:phero}
\begin{aligned}
	UpdatePheromones \rightarrow \tau_{ij}(t) = 
	(1-\rho)\cdot\tau(t-1) + \sum_{k=1}^{m}\bigtriangleup\tau_{ij}^k 
	\\
	\tau_{ij}^k = \frac{1}{L_k}, \qquad \text{if $arc(i,j)$ is used 
	by 
	ant k on its tour}
\end{aligned}
\end{equation}

\noindent
\rule{\textwidth}{0.4pt}
\footnotesize{Found an error? Let us know: \url{https://github.com/iridia-ulb/INFOH410/issues}}

\end{document} 
