\documentclass[11pt,a4paper]{article}
\usepackage[utf8]{inputenc}
\usepackage[T1]{fontenc}
\usepackage{amsthm} %numéroter les questions
\usepackage[english]{babel}
\usepackage{datetime}
\usepackage{xspace} % typographie IN
\usepackage{hyperref}% hyperliens
\usepackage[all]{hypcap} %lien pointe en haut des figures
%\usepackage[french]{varioref} %voir x p y

\usepackage{fancyhdr}% en têtes
\usepackage{graphicx} %include pictures
\usepackage{pgfplots}

\usepackage{tikz}
\usetikzlibrary{calc}
\usetikzlibrary{babel}
\usepackage{circuitikz}
% \usepackage{gnuplottex}
\usepackage{float}
\usepackage{ifthen}

\usepackage[top=1.3 in, bottom=1.3 in, left=1.3 in, right=1.3 in]{geometry}
\usepackage[]{pdfpages}
\usepackage[]{attachfile}

\usepackage{amsmath}
\usepackage{enumitem}
\setlist[enumerate]{label=\alph*)}% If you want only the x-th level to use this format, use '[enumerate,x]'
\usepackage{multirow}

\usepackage{aeguill} %guillemets


%%%%%%%%%%%%%%%%%%%%%%%%%%%%%%%%%%%%%%%%%%%%%%%%%%%%%%%%%%%%%%%%%%%%%%%%%%%%%%%%%%%%%%%%%%%
% READ THIS BEFORE CHANGING ANYTHING
%
%
% - TP number: can be changed using the command
%		\def \tpnumber {TP 1 }
%
% - Version: controlled by a new command:
%		\newcommand{\version}{v1.0.0}
%
% - Booleans: there are three booleans used in this document:
%	- 'corrige', controlled by defining the variable 'corrige'
% You can define those variables in a makefile using such a command:
% pdflatex -shell-escape -jobname="infoh410_tp1" "\def\corrige{} \input{infoh410_tp1.tex}"
%
%%%%%%%%%%%%%%%%%%%%%%%%%%%%%%%%%%%%%%%%%%%%%%%%%%%%%%%%%%%%%%%%%%%%%%%%%%%%%%%%%%%%%%%%%%%

%Numero du TP :
\def \tpnumber {TP PCFOL (Propositional Calculus and First Order Logic) }

\newcommand{\version}{v1.0.0}


% ########     #####      #####    ##         #########     ###     ##      ##  
% ##     ##  ##     ##  ##     ##  ##         ##           ## ##    ###     ##  
% ##     ##  ##     ##  ##     ##  ##         ##          ##   ##   ## ##   ##  
% ########   ##     ##  ##     ##  ##         ######     ##     ##  ##  ##  ##  
% ##     ##  ##     ##  ##     ##  ##         ##         #########  ##   ## ##  
% ##     ##  ##     ##  ##     ##  ##         ##         ##     ##  ##     ###  
% ########     #####      #####    #########  #########  ##     ##  ##      ##  
\newboolean{corrige}
\ifx\correction\undefined
\setboolean{corrige}{false}% pas de corrigé
\else
\setboolean{corrige}{true}%corrigé
\fi
% \setboolean{corrige}{true}% pas de corrigé


\definecolor{darkblue}{rgb}{0,0,0.5}

\pdfinfo{
/Author (ULB -- CoDe/IRIDIA)
/Title (\tpnumber, INFO-H-410)
/ModDate (D:\pdfdate)
}

\hypersetup{
pdftitle={\tpnumber [INFO-H-410] Techniques of AI},
pdfauthor={ULB -- CoDe/IRIDIA},
pdfsubject={}
}

\theoremstyle{definition}% questions pas en italique
\newtheorem{Q}{Question}[] % numéroter les questions [section] ou non []


%  ######     #####    ##       ##  ##       ##     ###     ##      ##  ######      #######   
% ##     ##  ##     ##  ###     ###  ###     ###    ## ##    ###     ##  ##    ##   ##     ##  
% ##         ##     ##  ## ## ## ##  ## ## ## ##   ##   ##   ## ##   ##  ##     ##  ##         
% ##         ##     ##  ##  ###  ##  ##  ###  ##  ##     ##  ##  ##  ##  ##     ##   #######   
% ##         ##     ##  ##       ##  ##       ##  #########  ##   ## ##  ##     ##         ##  
% ##     ##  ##     ##  ##       ##  ##       ##  ##     ##  ##     ###  ##    ##   ##     ##  
%  #######     #####    ##       ##  ##       ##  ##     ##  ##      ##  ######      #######   
\newcommand{\reponse}[1]{% pour intégrer une réponse : \reponse{texte} : sera inclus si \boolean{corrige}
\ifthenelse {\boolean{corrige}} {\paragraph{Answer:} \color{darkblue}   #1\color{black}} {}
}

\newcommand{\addcontentslinenono}[4]{\addtocontents{#1}{\protect\contentsline{#2}{#3}{#4}{}}}


%  #######   ##########  ##    ##   ##         #########  
% ##     ##      ##       ##  ##    ##         ##         
% ##             ##        ####     ##         ##         
%  #######       ##         ##      ##         ######     
%        ##      ##         ##      ##         ##         
% ##     ##      ##         ##      ##         ##         
%  #######       ##         ##      #########  #########  
%% fancy header & foot
\pagestyle{fancy}
\lhead{[INFO-H-410] Techniques of AI\\ \tpnumber}
\rhead{\version\\ page \thepage}
\chead{\ifthenelse{\boolean{corrige}}{Correction}{}}
%%

\setlength{\parskip}{0.2cm plus2mm minus1mm} %espacement entre §
\setlength{\parindent}{0pt}

% ##########  ########   ##########  ##         #########  
%     ##         ##          ##      ##         ##         
%     ##         ##          ##      ##         ##         
%     ##         ##          ##      ##         ######     
%     ##         ##          ##      ##         ##         
%     ##         ##          ##      ##         ##         
%     ##      ########       ##      #########  ######### 
\date{\vspace{-1.7cm}\version}
\title{\vspace{-2cm} \tpnumber \\ Techniques of AI [INFO-H-410] \ifthenelse{\boolean{corrige}}{~\\Correction}{}}


\begin{document}
\selectlanguage{english}

\maketitle

\fbox{
    \parbox{\textwidth}{
    \label{source}
\footnotesize{
Source files, code templates and corrections related to practical sessions can be found on the UV 
or on github (\url{https://github.com/iridia-ulb/INFOH410}).
}}}

\paragraph{Representation and Interpretation of Boolean Functions}
\begin{center}
\begin{tabular}{|l|l|}
\hline
\textbf{Symbol}            & \textbf{Name} \\ \hline
0                          & FALSE         \\ \hline
1                          & TRUE          \\ \hline
$!A$ / $\neg A$ & NOT A         \\ \hline
$A \wedge B$               & A AND B \\ \hline
$A \vee B$                 & A OR B \\ \hline
$A \oplus B$                 & A XOR B \\ \hline
\end{tabular}
\end{center}

\paragraph{Propositional Calculus}
\begin{Q}
Use truth tables to prove the following equivalences:
\begin{enumerate}
    \item $P \wedge (Q \vee R) \equiv (P \wedge Q) \vee (P \wedge R)$
    \item $P \vee (Q \wedge R) \equiv (P \vee Q) \wedge (P \vee R)$
    \item $\neg(P \wedge Q) \equiv \neg P \vee \neg Q$
    \item $\neg(P \vee Q) \equiv \neg P \wedge \neg Q$
    \item $P \Rightarrow Q \equiv \neg P \vee Q $
\end{enumerate}

\reponse{
\begin{enumerate}
    \item $P \wedge (Q \vee R) \equiv (P \wedge Q) \vee (P \wedge R)$

\begin{tabular}{|l|l|l|l|l|l|l|l|}
\hline
$P$ & $Q$ & $R$ & $(Q \vee R)$ & $(P \wedge Q)$ & $(P \wedge R)$ & $P \wedge (Q \vee R)$ & $(P \wedge Q) \vee (P \wedge R)$ \\ \hline
0   & 0   & 0   & 0            & 0              & 0              & 0                     & 0                                \\ \hline
0   & 0   & 1   & 1            & 0              & 0              & 0                     & 0                                \\ \hline
0   & 1   & 0   & 1            & 0              & 0              & 0                     & 0                                \\ \hline
0   & 1   & 1   & 1            & 0              & 0              & 0                     & 0                                \\ \hline
1   & 0   & 0   & 0            & 0              & 0              & 0                     & 0                                \\ \hline
1   & 0   & 1   & 1            & 0              & 1              & 1                     & 1                                \\ \hline
1   & 1   & 0   & 1            & 1              & 0              & 1                     & 1                                \\ \hline
1   & 1   & 1   & 1            & 1              & 1              & 1                     & 1                                \\ \hline
\end{tabular}
    \item $P \vee (Q \wedge R) \equiv (P \vee Q) \wedge (P \vee R)$

\begin{tabular}{|l|l|l|l|l|l|l|l|}
\hline
$P$ & $Q$ & $R$ & $(Q \wedge R)$ & $(P \vee Q)$ & $(P \vee R)$ & $P \vee (Q \wedge R)$ & $(P \vee Q) \wedge (P \vee R)$ \\ \hline
0   & 0   & 0   & 0              & 0            & 0            & 0                     & 0                              \\ \hline
0   & 0   & 1   & 0              & 0            & 1            & 0                     & 0                              \\ \hline
0   & 1   & 0   & 0              & 1            & 0            & 0                     & 0                              \\ \hline
0   & 1   & 1   & 1              & 1            & 1            & 1                     & 1                              \\ \hline
1   & 0   & 0   & 0              & 1            & 1            & 1                     & 1                              \\ \hline
1   & 0   & 1   & 1              & 1            & 1            & 1                     & 1                              \\ \hline
1   & 1   & 0   & 0              & 1            & 1            & 1                     & 1                              \\ \hline
1   & 1   & 1   & 1              & 1            & 1            & 1                     & 1                              \\ \hline
\end{tabular}

    \item $\neg(P \wedge Q) \equiv \neg P \vee \neg Q$

\begin{tabular}{|l|l|l|l|l|l|l|}
\hline
$P$ & $Q$ & $(P \wedge Q)$ & $\neg P$ & $\neg Q$ & $\neg (P \wedge Q)$ & $\neg P \vee \neg Q$ \\ \hline
0   & 0   & 0              & 1        & 1        & 1                   & 1                    \\ \hline
0   & 1   & 0              & 1        & 0        & 1                   & 1                    \\ \hline
1   & 0   & 0              & 0        & 1        & 1                   & 1                    \\ \hline
1   & 1   & 1              & 0        & 0        & 0                   & 0                    \\ \hline
\end{tabular}

    \item $\neg(P \vee Q) \equiv \neg P \wedge \neg Q$

\begin{tabular}{|l|l|l|l|l|l|l|}
\hline
$P$ & $Q$ & $(P \vee Q)$ & $\neg P$ & $\neg Q$ & $\neg (P \vee Q)$ & $\neg P \wedge \neg Q$ \\ \hline
0   & 0   & 0            & 1        & 1        & 1                 & 1                      \\ \hline
0   & 1   & 1            & 1        & 0        & 0                 & 0                      \\ \hline
1   & 0   & 1            & 0        & 1        & 0                 & 0                      \\ \hline
1   & 1   & 1            & 0        & 0        & 0                 & 0                      \\ \hline
\end{tabular}

    \item $P \Rightarrow Q \equiv \neg P \vee Q $

\begin{tabular}{|l|l|l|l|l|}
\hline
$P$ & $Q$ & $P \Rightarrow Q$ & $\neg P$ & $\neg P \vee Q$ \\ \hline
0   & 0   & 1                 & 1        & 1               \\ \hline
0   & 1   & 1                 & 1        & 1               \\ \hline
1   & 0   & 0                 & 0        & 0               \\ \hline
1   & 1   & 1                 & 0        & 1               \\ \hline
\end{tabular}

\end{enumerate}

}

\end{Q}

\begin{Q}
Formulate the following expressions as propositional sentences:
\begin{enumerate}
    \item If the unicorn is magical, then it is immortal.
    \item If the unicorn is not magical, then it is a mortal mammal.
    \item If the unicorn is either immortal or a mammal, then it is horned.
\end{enumerate}
Using truth tables, can you prove whether the unicorn is magical? Immortal? Horned?

\reponse{
    We use $G$ for magical, $O$ for mortal, $M$ for mammal and $H$ for horned.
\begin{enumerate}
    \item $G \Rightarrow \neg O$ 
    \item $\neg G \Rightarrow (O \wedge M) $
    \item $(\neg O \vee M) \Rightarrow H $
\end{enumerate}

% TODO finish
We can prove that the Unicorn is horned, but not whether it is magical, mortal nor mammal.
}
\end{Q}

\paragraph{First Order propositions}
\begin{Q}
Convert those expressions to first order logic expressions.
\begin{enumerate}
    \item All roads lead to Rome.
    \item All that glitters is not gold.
    \item The enemy of my enemy is my friend.
    \item A dog is a man's best friend.
\end{enumerate}

\reponse{
\begin{enumerate}
    \item $\forall x, Road(x) \Rightarrow GoToRome(x)$ 
    \item $\neg (\forall x, Glitters(x) \Rightarrow Gold(x) ) $
    \item $\forall x,y, Enemy(Me,x) \wedge Enemy(x,y) \Rightarrow Friend(Me,y) $
    \item $\forall x,y, Man(x) \wedge BestFriend(x,y) \Rightarrow Dog(y)$
\end{enumerate}


}
\end{Q}

\paragraph{Resolution}
\begin{Q}
    Prove the following using resolution (negate conclusion, convert to CNF, prove contradiction)
    \begin{enumerate}
        \item Given $KB = \{P \wedge Q\}$, prove that $KB \models P \vee Q$.
        \item Given $KB = \{P \vee Q, Q \Rightarrow (R \wedge S), (P \vee R) \Rightarrow U \}$, prove that $KB \models U$.

    \end{enumerate}

    \reponse{
    \begin{enumerate}
        \item Given $KB = \{P \wedge Q\}$, prove that $KB \models P \vee Q$.

            \begin{itemize}
                \item Negate conclusion: $\neg P \wedge \neg Q$
                \item Four sentences: $P, Q, \neg P, \neg Q$
                \item Resolve $P$ with $\neg P$, and $Q$ with $\neg Q$ gives \{\}. This means we have
                    a contradiction, or $KB \models P \vee Q$ is true.
            \end{itemize}

        \item Given $KB = \{P \vee Q, Q \Rightarrow (R \wedge S), (P \vee R) \Rightarrow U \}$, prove that $KB \models U$.
            \begin{itemize}
                \item Negate conclusion: $\neg U$.
                \item Convert to CNF: $KB = \{P \vee Q, (\neg Q \vee R) \wedge (\neg Q \vee S), (\neg P \vee U) \wedge (\neg R \vee U) \}$.
                \item 6 sentences.
                \item Resolve $\neg P \vee U$ with $\neg U$ which gives $\neg P$. $\neg R \vee U$ with $\neg U$ gives
                    $\neg R$.
                \item $\neg P \vee Q$ and $\neg P$ resolves in $Q$.
                \item $Q$ and $\neg Q \vee R$ resolves in $R$ which contradicts with $\neg R$ from the first
                    resolution. Thus, $KB \models U$ is true.
            \end{itemize}

    \end{enumerate}

    }
\end{Q}

\begin{Q}
    On the island of Knights and Knaves, everything a Knight says is true and everything a Knave says
    is false. You meet two people, Alice and Bob:
    \begin{itemize}
        \item Alice says "Neither Bob nor I are Knaves"
        \item Bob says "Alice is a Knave"
    \end{itemize}
    Using the proposition $A$ to represent "Alice is a Knight" ($\neg A$ means "Alice is a Knave")
    and $B$ to represent "Bob is a Knight".
    \begin{enumerate}
        \item Formulate what Alice and Bob said.
        \item Formulate that what they said is true if and only if they are knights.
        \item Put those into CNF form.
        \item Use resolution to prove who is what.
    \end{enumerate}
    \reponse{
    \begin{enumerate}
        \item Formulate what Alice and Bob said.

            Alice: $A \wedge B$, Bob: $\neg A$.
        \item Formulate that what they said is true if and only if they are knights.

            Alice: $A \Leftrightarrow A \wedge B$, Bob: $B \Leftrightarrow \neg A$
        \item Put those into CNF form.
            \begin{itemize}
                \item $A \Leftrightarrow A \wedge B$
                \item $A \Rightarrow A \wedge B$, $A \wedge B \Rightarrow A$
                \item $\neg A \vee (A \wedge B)$, $\neg(A \wedge B) \vee A$
                \item $\neg A \vee A$ (tautology), $\neg A \wedge B$, $\neg A \vee A \vee \neg B$(tautology)
                \item $ B \Leftrightarrow \neg A$
                \item $B \Rightarrow \neg A$, $\neg A \Rightarrow B$
                \item $\neg B \vee \neg A$, $A \vee B$

            \end{itemize}
        \item Use resolution to prove who is what.
            \begin{itemize}
                \item $\neg A \vee B$, $A \vee B$: $B$ (proof that Bob is a knight)
                \item Now we know that $B$ is true, so since $B \Rightarrow \neg A$: $\neg A$. (Alice
                    is a Knave)

            \end{itemize}
    \end{enumerate}
    }
\end{Q}


\begin{Q}
From "Sheep are animals", it follows that "The head of a sheep is the
head of an animal." Demonstrate that this inference is valid by
carrying out the following steps:
\begin{enumerate}
    \item Translate the premise and the conclusion into the language of first-order logic. Use three predicates: $H(h,x)$ (meaning "h is the head of x"), $S(x)$ ($Sheep(x)$), and $A(x)$ ($Animal(x)$).
    \item Negate the conclusion, and convert the premise and the negated conclusion into conjunctive normal form.
    \item Conclude.
\end{enumerate}

\reponse {

\begin{enumerate}
    \item $\forall x, S(x) \Rightarrow A(x)$, and $\forall x,y, H(y,z) \wedge S(z) \Rightarrow H(y,z) \wedge A(z)$

    \item Which translate to, once the arrows are transformed:
        $\forall x, \neg S(x) \vee A(x)$, and $\forall y,z, \neg (H(y,z) \wedge S(z) \vee (H(y,z) \wedge A(z) )$

        then we negate the conclusion:
        $\neg (\forall y,z, \neg (H(y,z) \wedge S(z) \vee (H(y,z) \wedge A(z) ))$
        
        switch to existential to move the negation:
        $\exists y,z,\neg( \neg (H(y,z) \wedge S(z) \vee (H(y,z) \wedge A(z) ))$
        
        Use De Morgan laws twice:
        $\exists y,z, H(y,z) \wedge S(z) \wedge (\neg H(y,z) \vee \neg A(z) )$

        Skolemize:
        $H(Y0,Z0) \wedge S(Z0) \wedge (\neg H(Y0,Z0) \vee \neg A(Z0))$

        Knowledge base of the problem:
        \begin{itemize}
            \item $H(Y0,Z0)$
            \item $S(Z0)$
            \item $\neg H(Y0,Z0) \vee \neg A(Z0) $
            \item $\neg S(x)  \vee A(x)$

        \end{itemize}

    \item Resolve $H(Y0,Z0)$ with $\neg H(Y0,Z0) \vee \neg A(Z0) $ gives $\neg A(Z0)$ 

        Then resolve $\neg S(x)  \vee A(x)$ with $\neg A(Z0)$ with unifier $\{ Z0/x \}$

        Resolve $\neg S(Z0)$ with $S(Z0)$ gives contradiction.






\end{enumerate}

}
\end{Q}

\noindent
\rule{\textwidth}{0.4pt}
\footnotesize{Found an error? Let us know: \url{https://github.com/iridia-ulb/INFOH410/issues}}

\end{document} 
