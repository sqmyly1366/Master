\documentclass[11pt,a4paper]{article}
\usepackage[utf8]{inputenc}
\usepackage[T1]{fontenc}
\usepackage{amsthm} %numéroter les questions
\usepackage[english]{babel}
\usepackage{datetime}
\usepackage{xspace} % typographie IN
\usepackage{hyperref}% hyperliens
\usepackage[all]{hypcap} %lien pointe en haut des figures
%\usepackage[french]{varioref} %voir x p y

\usepackage{fancyhdr}% en têtes
\usepackage{graphicx} %include pictures
\usepackage{pgfplots}

\usepackage{tikz}
\usetikzlibrary{calc}
\usetikzlibrary{babel}
\usepackage{circuitikz}
% \usepackage{gnuplottex}
\usepackage{float}
\usepackage{ifthen}

\usepackage[top=1.3 in, bottom=1.3 in, left=1.3 in, right=1.3 in]{geometry}
\usepackage[]{pdfpages}
\usepackage[]{attachfile}

\usepackage{amsmath}
\usepackage{enumitem}
\setlist[enumerate]{label=\alph*)}% If you want only the x-th level to use this format, use '[enumerate,x]'
\usepackage{multirow}

\usepackage{aeguill} %guillemets


%%%%%%%%%%%%%%%%%%%%%%%%%%%%%%%%%%%%%%%%%%%%%%%%%%%%%%%%%%%%%%%%%%%%%%%%%%%%%%%%%%%%%%%%%%%
% READ THIS BEFORE CHANGING ANYTHING
%
%
% - TP number: can be changed using the command
%		\def \tpnumber {TP 1 }
%
% - Version: controlled by a new command:
%		\newcommand{\version}{v1.0.0}
%
% - Booleans: there are three booleans used in this document:
%	- 'corrige', controlled by defining the variable 'corrige'
% You can define those variables in a makefile using such a command:
% pdflatex -shell-escape -jobname="infoh410_tp1" "\def\corrige{} \input{infoh410_tp1.tex}"
%
%%%%%%%%%%%%%%%%%%%%%%%%%%%%%%%%%%%%%%%%%%%%%%%%%%%%%%%%%%%%%%%%%%%%%%%%%%%%%%%%%%%%%%%%%%%

%Numero du TP :
\def \tpnumber {TP C (Concept Learning) }

\newcommand{\version}{v1.0.0}


% ########     #####      #####    ##         #########     ###     ##      ##  
% ##     ##  ##     ##  ##     ##  ##         ##           ## ##    ###     ##  
% ##     ##  ##     ##  ##     ##  ##         ##          ##   ##   ## ##   ##  
% ########   ##     ##  ##     ##  ##         ######     ##     ##  ##  ##  ##  
% ##     ##  ##     ##  ##     ##  ##         ##         #########  ##   ## ##  
% ##     ##  ##     ##  ##     ##  ##         ##         ##     ##  ##     ###  
% ########     #####      #####    #########  #########  ##     ##  ##      ##  
\newboolean{corrige}
\ifx\correction\undefined
\setboolean{corrige}{false}% pas de corrigé
\else
\setboolean{corrige}{true}%corrigé
\fi

% \setboolean{corrige}{true}% pas de corrigé

\definecolor{darkblue}{rgb}{0,0,0.5}

\pdfinfo{
/Author (ULB -- CoDe/IRIDIA)
/Title (\tpnumber, INFO-H-410)
/ModDate (D:\pdfdate)
}

\hypersetup{
pdftitle={\tpnumber [INFO-H-410] Techniques of AI},
pdfauthor={ULB -- CoDe/IRIDIA},
pdfsubject={}
}

\theoremstyle{definition}% questions pas en italique
\newtheorem{Q}{Question}[] % numéroter les questions [section] ou non []


%  ######     #####    ##       ##  ##       ##     ###     ##      ##  ######      #######   
% ##     ##  ##     ##  ###     ###  ###     ###    ## ##    ###     ##  ##    ##   ##     ##  
% ##         ##     ##  ## ## ## ##  ## ## ## ##   ##   ##   ## ##   ##  ##     ##  ##         
% ##         ##     ##  ##  ###  ##  ##  ###  ##  ##     ##  ##  ##  ##  ##     ##   #######   
% ##         ##     ##  ##       ##  ##       ##  #########  ##   ## ##  ##     ##         ##  
% ##     ##  ##     ##  ##       ##  ##       ##  ##     ##  ##     ###  ##    ##   ##     ##  
%  #######     #####    ##       ##  ##       ##  ##     ##  ##      ##  ######      #######   
\newcommand{\reponse}[1]{% pour intégrer une réponse : \reponse{texte} : sera inclus si \boolean{corrige}
\ifthenelse {\boolean{corrige}} {\paragraph{Answer:} \color{darkblue}   #1\color{black}} {}
}

\newcommand{\addcontentslinenono}[4]{\addtocontents{#1}{\protect\contentsline{#2}{#3}{#4}{}}}


%  #######   ##########  ##    ##   ##         #########  
% ##     ##      ##       ##  ##    ##         ##         
% ##             ##        ####     ##         ##         
%  #######       ##         ##      ##         ######     
%        ##      ##         ##      ##         ##         
% ##     ##      ##         ##      ##         ##         
%  #######       ##         ##      #########  #########  
%% fancy header & foot
\pagestyle{fancy}
\lhead{[INFO-H-410] Techniques of AI\\ \tpnumber}
\rhead{\version\\ page \thepage}
\chead{\ifthenelse{\boolean{corrige}}{Correction}{}}
%%

\setlength{\parskip}{0.2cm plus2mm minus1mm} %espacement entre §
\setlength{\parindent}{0pt}

% ##########  ########   ##########  ##         #########  
%     ##         ##          ##      ##         ##         
%     ##         ##          ##      ##         ##         
%     ##         ##          ##      ##         ######     
%     ##         ##          ##      ##         ##         
%     ##         ##          ##      ##         ##         
%     ##      ########       ##      #########  ######### 
\date{\vspace{-1.7cm}\version}
\title{\vspace{-2cm} \tpnumber \\ Techniques of AI [INFO-H-410] \ifthenelse{\boolean{corrige}}{~\\Correction}{}}


\begin{document}
\selectlanguage{english}

\maketitle

\fbox{
    \parbox{\textwidth}{
    \label{source}
\footnotesize{
Source files, code templates and corrections related to practical sessions can be found on the UV 
or on github (\url{https://github.com/iridia-ulb/INFOH410}).
}}}

\paragraph{Calculating the size of the hypothesis space}
\begin{Q}
    Suppose there are m attributes in a learning task and that every attribute $i$ can take $k_i$
possible values. What will be the size of the hypothesis space?
\reponse{
    syntactically different: $(k_i+2)^m$ (+2 accounts for empty and don't care),
    semantically different: $1+(k_i+1)^m$ (full empty attribute + attributes with don't care)
}
\end{Q}

\paragraph{Order of training instances}
\begin{Q}
    In candidate elimination, suppose you have $n$ training instances $T_1...T_n$. After the $n_{th}$
training instance, candidate elimination learned the boundaries $S$ and $G$. Will $S$ and $G$
differ or not when providing the training instances in reverse order: $T_n...T_1$? Explain why
(not).
\reponse{
   The concept of version space aims at invariance to instance order, keeping not a
single concept description but a set of possible descriptions that evolves as new instances
are presented, so order does not matter.
}
\end{Q}


\begin{Q}
What is the version space while tracing the candidate elimination algorithm with the
following examples?

$Architecture \in \{Gothic, Romanesque\}$ \\
$Size \in \{Small, Large\}$ \\
$Steeples \in \{Zero, One, Two\}$ \\
\begin{center}
\begin{tabular}{|l|l|l|l|}
\hline
\textbf{Architecture} & \textbf{Size} & \textbf{Steeples} & \textbf{Classified ?} \\ \hline
G                     & S             & 2                 & True                  \\ \hline
R                     & S             & 2                 & False                 \\ \hline
G                     & L             & 2                 & True                  \\ \hline
G                     & S             & 0                 & False                 \\ \hline
R                     & L             & 2                 & True                  \\ \hline
\end{tabular}
\end{center}


\reponse{
$S_0 = \{\emptyset, \emptyset, \emptyset\}$ and $G0 = \{?, ?, ?\}$\\
$S_1 = \{G, S, 2\}$ and $G_1 = \{?, ?, ?\}$\\
$S_2 = \{G, S, 2\}$ and $G_2 = \{G, ?, ?\}$\\
$S_3 = \{G, ?, 2\}$ and $G_3 = \{G, ?, ?\}$\\
$S_4 = \{G, ?, 2\}$ and $S_4 = \{G, ?, 2\}$\\
$S_5 = G_5 = \emptyset$\\

}
\end{Q}

\paragraph{Rectangular version spaces and candidate elimination}
\begin{Q}
    Consider the instance space consisting of integer points in the $x, y$ plane and the set of
hypotheses H consisting of rectangles. More precisely, hypotheses are of the form $a \leq x \leq b$,
$c \leq y \leq d$, where $a$, $b$, $c$ and $d$ can be any integers. Consider the version space with 
respect to the set of positive (+) and negative (-) training examples:

\begin{tabular}{ll}
$-(1,3)$ & $-(2,6)$ \\
$+(6,5)$ & $+(5,3)$ \\
$-(9,4)$ & $-(5,1)$ \\
$+(4,4)$ & $-(5,8)$
\end{tabular}

\begin{enumerate}
    \item What is the S boundary of the version space in this case? Write a diagram with the
        training data and the S boundary.
    \item What is the G boundary of this version space? Draw that in the diagram as well.
    \item Use python to find the S and the G boundary and plot them 
        (you can use the code template (see page~\pageref{source})).
    \item Suppose the learner may suggest a new $x, y$ instance and ask the trainer for its 
        classification. Suggest a query guaranteed to reduce the size of the version space, 
        regardless how the trainer classifies it. Suggest one that will not.
    \item Now assume you are the teacher, attempting to reach a particular target concept,
$3 \leq x \leq 5, 2 \leq y \leq 9$. What is the smallest number of training examples you can
provide so that the Candidate- Elimination algorithm will perfectly learn the concept?

\end{enumerate}

\reponse{
(a) $S = \{4,6,3,5\}$\\
(b) $G = \{3,8,2,7\}$\\
(c) see github for implementation.\\
(d) To reduce the VS: $(4, 6)$ or $(7, 3)$, instances with no impact on the VS: $(5, 4)$ or $(3, 9)$.\\
(e) 6 points: +$(3, 9)$, +$(5, 2)$, -$(2, 5)$, -$(4, 1)$, -$(6, 5)$, -$(4, 10)$.
}

\end{Q}

\paragraph{Finding a maximally specific consistent hypothesis}
\begin{Q}
Consider a concept learning problem in which each instance is a real number and in which
each hypothesis is an interval over the reals. More precisely, each hypothesis in H is of the
form: $a < x < b$ as in $4.5 < x < 6.1$, meaning that all real numbers between 4.5 and 6.1 are
classified as positive examples and all others are classified as negative examples.
\begin{enumerate}
    \item Explain informally why there cannot be a maximally specific consistent hypothesis for
any set of positive training examples.
    \item Suggest a modification to the hypothesis representation so this will not happen.
\end{enumerate}

\reponse{
    (a) $a < x < b$ is not a well defined hypothesis representation,\\
    (b) $a \leq x \leq b$
}
\end{Q}


\noindent
\rule{\textwidth}{0.4pt}
\footnotesize{Found an error? Let us know: \url{https://github.com/iridia-ulb/INFOH410/issues}}

\end{document} 
