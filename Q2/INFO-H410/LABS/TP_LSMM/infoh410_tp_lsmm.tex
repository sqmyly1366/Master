\documentclass[11pt,a4paper]{article}
\usepackage[utf8]{inputenc}
\usepackage[T1]{fontenc}
\usepackage{amsthm} %numéroter les questions
\usepackage[english]{babel}
\usepackage{datetime}
\usepackage{xspace} % typographie IN
\usepackage{hyperref}% hyperliens
\usepackage[all]{hypcap} %lien pointe en haut des figures

\usepackage{fancyhdr}% en têtes
\usepackage{graphicx} %include pictures
\usepackage{pgfplots}

\usepackage{tikz}
\usetikzlibrary{calc}
\usetikzlibrary{babel}
\usepackage{circuitikz}
% \usepackage{gnuplottex}
\usepackage{float}
\usepackage{ifthen}

\usepackage[top=1.3 in, bottom=1.3 in, left=1.3 in, right=1.3 in]{geometry}
\usepackage[]{pdfpages}
\usepackage[]{attachfile}

\usepackage{amsmath}
\usepackage{enumitem}
\setlist[enumerate]{label=\alph*)}% If you want only the x-th level to use this format, use '[enumerate,x]'
\usepackage{multirow}

\usepackage{chessboard}
\storechessboardstyle{4x4}{maxfield=d4}

%%%%%%%%%%%%%%%%%%%%%%%%
% Algorithms
%%%%%%%%%%%%%%%%%%%%%%%%
\usepackage{algorithm}
\usepackage[noend]{algpseudocode} % From algorithmicx

%%%%%%%%%%%%
% Vertical line trick in algorithms, https://tex.stackexchange.com/questions/292815/how-can-i-create-vertical-lines-indentation-in-algorithm-pseudo-code-correctly-w/292838#292838
\makeatletter
% start with some helper code
% This is the vertical rule that is inserted
\newcommand*{\algrule}[1][\algorithmicindent]{%
  \makebox[#1][l]{%
    \hspace*{.2em}% <------------- This is where the rule starts from
    \color{black!10}\vrule height .75\baselineskip depth .25\baselineskip
  }
}

\newcount\ALG@printindent@tempcnta
\def\ALG@printindent{%
    \ifnum \theALG@nested>0% is there anything to print
    \ifx\ALG@text\ALG@x@notext% is this an end group without any text?
    % do nothing
    \else
    \unskip
    % draw a rule for each indent level
    \ALG@printindent@tempcnta=1
    \loop
    \algrule[\csname ALG@ind@\the\ALG@printindent@tempcnta\endcsname]%
    \advance \ALG@printindent@tempcnta 1
    \ifnum \ALG@printindent@tempcnta<\numexpr\theALG@nested+1\relax
    \repeat
    \fi
    \fi
}
% the following line injects our new indent handling code in place of the default spacing
\patchcmd{\ALG@doentity}{\noindent\hskip\ALG@tlm}{\ALG@printindent}{}{\errmessage{failed to patch}}
\patchcmd{\ALG@doentity}{\item[]\nointerlineskip}{}{}{} % no spurious vertical space
% end vertical rule patch for algorithmicx
\makeatother
%
%%%%%%%%%%%%


%%%%%%%%%%%%%%%%%%%%%%%%%%%%%%%%%%%%%%%%%%%%%%%%%%%%%%%%%%%%%%%%%%%%%%%%%%%%%%%%%%%%%%%%%%%
% READ THIS BEFORE CHANGING ANYTHING
%
%
% - TP number: can be changed using the command
%		\def \tpnumber {TP 1 }
%
% - Version: controlled by a new command:
%		\newcommand{\version}{v1.0.0}
%
% - Booleans: there are three booleans used in this document:
%	- 'corrige', controlled by defining the variable 'corrige'
% You can define those variables in a makefile using such a command:
% pdflatex -shell-escape -jobname="infoh410_tp1" "\def\corrige{} \input{infoh410_tp1.tex}"
%
%%%%%%%%%%%%%%%%%%%%%%%%%%%%%%%%%%%%%%%%%%%%%%%%%%%%%%%%%%%%%%%%%%%%%%%%%%%%%%%%%%%%%%%%%%%

%Numero du TP :
\def \tpnumber {TP LSMM (Local Search and Min-Max algorithm) }

\newcommand{\version}{v1.0.0}


% ########     #####      #####    ##         #########     ###     ##      ##  
% ##     ##  ##     ##  ##     ##  ##         ##           ## ##    ###     ##  
% ##     ##  ##     ##  ##     ##  ##         ##          ##   ##   ## ##   ##  
% ########   ##     ##  ##     ##  ##         ######     ##     ##  ##  ##  ##  
% ##     ##  ##     ##  ##     ##  ##         ##         #########  ##   ## ##  
% ##     ##  ##     ##  ##     ##  ##         ##         ##     ##  ##     ###  
% ########     #####      #####    #########  #########  ##     ##  ##      ##  
\newboolean{corrige}
\ifx\correction\undefined
\setboolean{corrige}{false}% pas de corrigé
\else
\setboolean{corrige}{true}%corrigé
\fi
% \setboolean{corrige}{true}% pas de corrigé


\definecolor{darkblue}{rgb}{0,0,0.5}

\pdfinfo{
/Author (ULB -- CoDe/IRIDIA)
/Title (\tpnumber, INFO-H-410)
/ModDate (D:\pdfdate)
}

\hypersetup{
pdftitle={\tpnumber [INFO-H-410] Techniques of AI},
pdfauthor={ULB -- CoDe/IRIDIA},
pdfsubject={}
}

\theoremstyle{definition}% questions pas en italique
\newtheorem{Q}{Question}[] % numéroter les questions [section] ou non []


%  ######     #####    ##       ##  ##       ##     ###     ##      ##  ######      #######   
% ##     ##  ##     ##  ###     ###  ###     ###    ## ##    ###     ##  ##    ##   ##     ##  
% ##         ##     ##  ## ## ## ##  ## ## ## ##   ##   ##   ## ##   ##  ##     ##  ##         
% ##         ##     ##  ##  ###  ##  ##  ###  ##  ##     ##  ##  ##  ##  ##     ##   #######   
% ##         ##     ##  ##       ##  ##       ##  #########  ##   ## ##  ##     ##         ##  
% ##     ##  ##     ##  ##       ##  ##       ##  ##     ##  ##     ###  ##    ##   ##     ##  
%  #######     #####    ##       ##  ##       ##  ##     ##  ##      ##  ######      #######   
\newcommand{\reponse}[1]{% pour intégrer une réponse : \reponse{texte} : sera inclus si \boolean{corrige}
\ifthenelse {\boolean{corrige}} {\paragraph{Answer:} \color{darkblue}   #1\color{black}} {}
}

\newcommand{\addcontentslinenono}[4]{\addtocontents{#1}{\protect\contentsline{#2}{#3}{#4}{}}}


%  #######   ##########  ##    ##   ##         #########  
% ##     ##      ##       ##  ##    ##         ##         
% ##             ##        ####     ##         ##         
%  #######       ##         ##      ##         ######     
%        ##      ##         ##      ##         ##         
% ##     ##      ##         ##      ##         ##         
%  #######       ##         ##      #########  #########  
%% fancy header & foot
\pagestyle{fancy}
\lhead{[INFO-H-410] Techniques of AI\\ \tpnumber}
\rhead{\version\\ page \thepage}
\chead{\ifthenelse{\boolean{corrige}}{Correction}{}}
%%

\setlength{\parskip}{0.2cm plus2mm minus1mm} %espacement entre §
\setlength{\parindent}{0pt}

% ##########  ########   ##########  ##         #########  
%     ##         ##          ##      ##         ##         
%     ##         ##          ##      ##         ##         
%     ##         ##          ##      ##         ######     
%     ##         ##          ##      ##         ##         
%     ##         ##          ##      ##         ##         
%     ##      ########       ##      #########  ######### 
\date{\vspace{-1.7cm}\version}
\title{\vspace{-2cm} \tpnumber \\ Techniques of AI [INFO-H-410] \ifthenelse{\boolean{corrige}}{~\\Correction}{}}


\begin{document}
\selectlanguage{english}

\maketitle

\fbox{
    \parbox{\textwidth}{
    \label{source}
\footnotesize{
Source files, code templates and corrections related to practical sessions can be found on the UV 
or on github (\url{https://github.com/iridia-ulb/INFOH410}).
}}}

\paragraph{Hill Climbing}
\begin{Q}
We want to solve the $N$-queens problem. Suppose you want to place $N$ queens of a  $N\times N$ 
chess board so that no queen can attack another queen.

\chessboard[style=4x4,setblack={Qa4,Qb3,Qc4,Qd3},showmover=false]%&
\chessboard[style=4x4,setblack={Qa4,Qb3,Qc4,Qd2},showmover=false]%&
\chessboard[style=4x4,setblack={Qa4,Qb1,Qc4,Qd2},showmover=false]%&
\chessboard[style=4x4,setblack={Qa3,Qb1,Qc4,Qd2},showmover=false] \\

\begin{enumerate}
    \item What would be a good heuristic for this problem ? Calculate this heuristic for each of the above board.
    \item How many board arrangments are there ?
    \item What would be a good state representation for this puzzle ? 
    \item Given a clever state representation, how many board arrangments are there ?
 \end{enumerate}
\reponse{
    (a) The number of pairs of attacking queens, h = 5, h = 3, h = 1, h = 0.\\
    (b) There are 16 squares, $16 * 15 * 14 * 13 = 43680$.\\
    (c) 4 digits, one digit represents the vertical position of a given queen on a row, using this
    state representation, already skrinks the space of solution since no 2 queens can be in the same
    column. With this state representation, the first board is <0,1,0,1>, the second <0,1,0,2>, the third
    <0,3,0,2>.\\
    (d) If we say that we only want the permutations of 4 different digits from 0 to N (that
    way we already discard any position where 2 queens are in the same row or column) then there
    are $N! = 24$ arrangments.

}
\end{Q}

\begin{Q}
    Implement exhaustive search in python (you can use the template (see page~\pageref{source})) for the aforementioned puzzle.
\reponse{
    see github for implementation
}
\end{Q}


\begin{Q}
Implement (stochastic) Hill Climbing local search in python for this puzzle. Now compare your results with 
exhaustive search, what happens if N = 10, 15, 20 ? Why ?
\reponse{
    see github for implementation, exhaustive cannot work with N=12, because the search space is
    too big ($N!$), whereas hill climbing works because the smaller neighborhood makes the branching
    factor acceptable.
}
\end{Q}


\paragraph{The Minimax algorithm}

\begin{Q}
    Perform the minimax algorithm on the following tree, first without and then with $\alpha \beta$-pruning.
\begin{center}
\begin{tikzpicture}[scale=1.5,font=\footnotesize]
\tikzstyle{solid node}=[circle,draw,inner sep=1.5,fill=black]
\tikzstyle{hollow node}=[circle,draw,inner sep=1.5]
\tikzstyle{level 1}=[level distance=10mm,sibling distance=4.5cm]
\tikzstyle{level 2}=[level distance=10mm,sibling distance=1.5cm]
\tikzstyle{level 3}=[level distance=10mm,sibling distance=0.7cm]
\tikzstyle{level 4}=[level distance=10mm,sibling distance=0.3cm]
\node(0)[solid node,label=left:{MAX}]{}
child{node[hollow node,label=left:{MIN}]{}
child{node[solid node,label=left:{MAX}]{}
    child{node[hollow node,label=left:{MIN}]{} 
        child{node[solid node,label=below:{4}]{} edge from parent node[left]{}}
        child{node[solid node,label=below:{3}]{} edge from parent node[left]{}}
        child{node[solid node,label=below:{5}]{} edge from parent node[left]{}}
    }
    child{node[hollow node,label=left:{}]{}
        child{node[solid node,label=below:{2}]{} edge from parent node[left]{}}
        child{node[solid node,label=below:{1}]{} edge from parent node[left]{}}
    }
}
child{node[solid node]{}{}
    child{node[hollow node,label=left:{}]{} 
        child{node[solid node,label=below:{4}]{} edge from parent node[left]{}}
        child{node[solid node,label=below:{2}]{} edge from parent node[left]{}}
        child{node[solid node,label=below:{3}]{} edge from parent node[left]{}}
    }
}
child{node[solid node]{}
    child{node[hollow node,label=left:{}]{} 
        child{node[solid node,label=below:{5}]{} edge from parent node[left]{}}
        child{node[solid node,label=below:{4}]{} edge from parent node[left]{}}
    }
    child{node[hollow node,label=left:{}]{} 
        child{node[solid node,label=below:{7}]{} edge from parent node[left]{}}
    }
    child{node[hollow node,label=left:{}]{}
        child{node[solid node,label=below:{3}]{} edge from parent node[left]{}}
        child{node[solid node,label=below:{2}]{} edge from parent node[left]{}}
    }
}
}
child{node[hollow node,label=above right:{}]{}
child{node[solid node]{}
    child{node[hollow node,label=left:{}]{} 
        child{node[solid node,label=below:{1}]{} edge from parent node[left]{}}
        child{node[solid node,label=below:{4}]{} edge from parent node[left]{}}
        child{node[solid node,label=below:{0}]{} edge from parent node[left]{}}
    }
}
child{node[solid node]{}
    child{node[hollow node,label=left:{}]{} 
        child{node[solid node,label=below:{5}]{} edge from parent node[left]{}}
        child{node[solid node,label=below:{3}]{} edge from parent node[left]{}}
    }
    child{node[hollow node,label=left:{}]{} 
        child{node[solid node,label=below:{0}]{} edge from parent node[left]{}}
    }
}
child{node[solid node]{}
    child{node[hollow node,label=left:{}]{} 
        child{node[solid node,label=below:{2}]{} edge from parent node[left]{}}
        child{node[solid node,label=below:{7}]{} edge from parent node[left]{}}
        child{node[solid node,label=below:{4}]{} edge from parent node[left]{}}
    }
    child{node[hollow node,label=left:{}]{} 
        child{node[solid node,label=below:{3}]{} edge from parent node[left]{}}
        child{node[solid node,label=below:{6}]{} edge from parent node[left]{}}
    }
    child{node[hollow node,label=left:{}]{}
        child{node[solid node,label=below:{5}]{} edge from parent node[left]{}}
        child{node[solid node,label=below:{3}]{} edge from parent node[left]{}}
        child{node[solid node,label=below:{1}]{} edge from parent node[left]{}}
    }
}
};
\end{tikzpicture}
\end{center}

\reponse{
With regular minimax:\\
\begin{tikzpicture}[scale=1.5,font=\footnotesize]
\tikzstyle{solid node}=[circle,draw,inner sep=1.5,fill=black]
\tikzstyle{hollow node}=[circle,draw,inner sep=1.5]
\tikzstyle{level 1}=[level distance=10mm,sibling distance=4.5cm]
\tikzstyle{level 2}=[level distance=10mm,sibling distance=1.5cm]
\tikzstyle{level 3}=[level distance=10mm,sibling distance=0.7cm]
\tikzstyle{level 4}=[level distance=10mm,sibling distance=0.3cm]
\node(0)[solid node,label=left:{MAX (2)}]{}
    child{node[hollow node,label=left:{MIN (2)}]{}
    child{node[solid node,label=left:{MAX (3)}]{}
        child{node[hollow node,label=left:{MIN (3)}]{} 
        child{node[solid node,label=below:{4}]{} edge from parent node[left]{}}
        child{node[solid node,label=below:{3}]{} edge from parent node[left]{}}
        child{node[solid node,label=below:{5}]{} edge from parent node[left]{}}
    }
    child{node[hollow node,label=left:{(1)}]{}
        child{node[solid node,label=below:{2}]{} edge from parent node[left]{}}
        child{node[solid node,label=below:{1}]{} edge from parent node[left]{}}
    }
}
child{node[solid node,label=left:{(2)} ]{}
    child{node[hollow node,label=left:{(2)}]{} 
        child{node[solid node,label=below:{4}]{} edge from parent node[left]{}}
        child{node[solid node,label=below:{2}]{} edge from parent node[left]{}}
        child{node[solid node,label=below:{3}]{} edge from parent node[left]{}}
    }
}
child{node[solid node,label=left:{(7)}]{}
    child{node[hollow node,label=left:{(4)}]{} 
        child{node[solid node,label=below:{5}]{} edge from parent node[left]{}}
        child{node[solid node,label=below:{4}]{} edge from parent node[left]{}}
    }
    child{node[hollow node,label=left:{(7)}]{} 
        child{node[solid node,label=below:{7}]{} edge from parent node[left]{}}
    }
    child{node[hollow node,label=left:{(2)}]{}
        child{node[solid node,label=below:{3}]{} edge from parent node[left]{}}
        child{node[solid node,label=below:{2}]{} edge from parent node[left]{}}
    }
}
}
child{node[hollow node,label=above right:{(0)}]{}
    child{node[solid node,label=left:{(0)}]{}
        child{node[hollow node,label=left:{(0)}]{} 
        child{node[solid node,label=below:{1}]{} edge from parent node[left]{}}
        child{node[solid node,label=below:{4}]{} edge from parent node[left]{}}
        child{node[solid node,label=below:{0}]{} edge from parent node[left]{}}
    }
}
child{node[solid node,label=left:{(3)}]{}
    child{node[hollow node,label=left:{(3)}]{} 
        child{node[solid node,label=below:{5}]{} edge from parent node[left]{}}
        child{node[solid node,label=below:{3}]{} edge from parent node[left]{}}
    }
    child{node[hollow node,label=left:{(0)}]{} 
        child{node[solid node,label=below:{0}]{} edge from parent node[left]{}}
    }
}
child{node[solid node,label=left:{(3)}]{}
    child{node[hollow node,label=left:{(2)}]{} 
        child{node[solid node,label=below:{2}]{} edge from parent node[left]{}}
        child{node[solid node,label=below:{7}]{} edge from parent node[left]{}}
        child{node[solid node,label=below:{4}]{} edge from parent node[left]{}}
    }
    child{node[hollow node,label=left:{(3)}]{} 
        child{node[solid node,label=below:{3}]{} edge from parent node[left]{}}
        child{node[solid node,label=below:{6}]{} edge from parent node[left]{}}
    }
    child{node[hollow node,label=left:{(1)}]{}
        child{node[solid node,label=below:{5}]{} edge from parent node[left]{}}
        child{node[solid node,label=below:{3}]{} edge from parent node[left]{}}
        child{node[solid node,label=below:{1}]{} edge from parent node[left]{}}
    }
}
};
\end{tikzpicture}
$\alpha \beta$-pruning saves 17 evaluations, unvisited nodes are labelled in red

\begin{tikzpicture}[scale=1.5,font=\footnotesize]
\tikzstyle{solid node}=[circle,draw,inner sep=1.5,fill=black]
\tikzstyle{hollow node}=[circle,draw,inner sep=1.5]
\tikzstyle{level 1}=[level distance=10mm,sibling distance=4.5cm]
\tikzstyle{level 2}=[level distance=10mm,sibling distance=1.5cm]
\tikzstyle{level 3}=[level distance=10mm,sibling distance=0.7cm]
\tikzstyle{level 4}=[level distance=10mm,sibling distance=0.3cm]
\node(0)[solid node,label=left:{MAX (2)}]{}
    child{node[hollow node,label=left:{MIN (2)}]{}
    child{node[solid node,label=left:{MAX (3)}]{}
        child{node[hollow node,label=left:{MIN (3)}]{} 
        child{node[solid node,label=below:{4}]{} edge from parent node[left]{}}
        child{node[solid node,label=below:{3}]{} edge from parent node[left]{}}
        child{node[solid node,label=below:{5}]{} edge from parent node[left]{}}
    }
    child{node[hollow node,label=left:{(2)}]{}
        child{node[solid node,label=below:{2}]{} edge from parent node[left]{}}
        child{node[solid node,label=below:{1},color=red]{} edge from parent node[left]{}}
    }
}
child{node[solid node,label=left:{(2)} ]{}
    child{node[hollow node,label=left:{(2)}]{} 
        child{node[solid node,label=below:{4}]{} edge from parent node[left]{}}
        child{node[solid node,label=below:{2}]{} edge from parent node[left]{}}
        child{node[solid node,label=below:{3}]{} edge from parent node[left]{}}
    }
}
child{node[solid node,label=left:{(4)}]{}
    child{node[hollow node,label=left:{(4)}]{} 
        child{node[solid node,label=below:{5}]{} edge from parent node[left]{}}
        child{node[solid node,label=below:{4}]{} edge from parent node[left]{}}
    }
    child{node[hollow node,label=left:{},color=red]{} 
        child{node[solid node,label=below:{7},color=red]{} edge from parent node[left]{}}
    }
    child{node[hollow node,label=left:{},color=red]{}
        child{node[solid node,label=below:{3},color=red]{} edge from parent node[left]{}}
        child{node[solid node,label=below:{2},color=red]{} edge from parent node[left]{}}
    }
}
}
child{node[hollow node,label=above right:{(1)}]{}
    child{node[solid node,label=left:{(1)}]{}
        child{node[hollow node,label=left:{(1)}]{} 
        child{node[solid node,label=below:{1}]{} edge from parent node[left]{}}
        child{node[solid node,label=below:{4},color=red]{} edge from parent node[left]{}}
        child{node[solid node,label=below:{0},color=red]{} edge from parent node[left]{}}
    }
}
child{node[solid node,label=left:{},color=red]{}
    child{node[hollow node,label=left:{},color=red]{} 
        child{node[solid node,label=below:{5},color=red]{} edge from parent node[left]{}}
        child{node[solid node,label=below:{3},color=red]{} edge from parent node[left]{}}
    }
    child{node[hollow node,label=left:{},color=red]{} 
        child{node[solid node,label=below:{0},color=red]{} edge from parent node[left]{}}
    }
}
child{node[solid node,label=left:{},color=red]{}
    child{node[hollow node,label=left:{},color=red]{} 
        child{node[solid node,label=below:{2},color=red]{} edge from parent node[left]{}}
        child{node[solid node,label=below:{7},color=red]{} edge from parent node[left]{}}
        child{node[solid node,label=below:{4},color=red]{} edge from parent node[left]{}}
    }
    child{node[hollow node,label=left:{},color=red]{} 
        child{node[solid node,label=below:{3},color=red]{} edge from parent node[left]{}}
        child{node[solid node,label=below:{6},color=red]{} edge from parent node[left]{}}
    }
    child{node[hollow node,label=left:{},color=red]{}
        child{node[solid node,label=below:{5},color=red]{} edge from parent node[left]{}}
        child{node[solid node,label=below:{3},color=red]{} edge from parent node[left]{}}
        child{node[solid node,label=below:{1},color=red]{} edge from parent node[left]{}}
    }
}
};
\end{tikzpicture}

}
\end{Q}

\begin{Q}
Implement the tic tac toe game using python so that you can play against an AI using the
minimax algorithm (you can use the template (see page~\pageref{source})).
\begin{center}
\begin{tabular}{c|c|c}
  X & X & O \\      \hline
  O & X & X \\      \hline
  O & X & O
\end{tabular}
\end{center}
\reponse{
    see github for implementation
}
\end{Q}




% ##   ##    ##         
% ##  ##     ##         
% ## ##      ##         
% ####       ##         
% ##  ##     ##         
% ##   ##    ##         
% ##    ##   ######### 

\paragraph{Local search applied to graphs: Kernighan-Lin}~\newline

Kernighan-Lin is an algorithm to split a graph $\mathcal{G} = (\mathcal{V}, \mathcal{E})$ into two partitions $P_1$ and $P_2$ with a minimal cutisze. The cutsize $c$ is computed as $\sum_{i \in P_1, j \in P_2} w(e_{i,j})$, where $e_{i,j}$ is an edge $\in \mathcal{E}$ connecting two vertices $i, j \in \mathcal{V}$ in different partitions and $w$ the weight of said edge.

The algorithm will be swapping pairs of vertices across the partition boundary in order to gradually improve the cutsize, as described in aglo~\ref{algo:kl}.
The costs and gains it refers to are computed as follows:
\begin{itemize}
    \item External cost of vertex $x \in P_1$: $E_x = \sum_{i\in P_2}w(e_{x,i})$
    \item Internal cost of vertex $x \in P_1$: $I_x = \sum_{i\in P_1}w(e_{x,i})$
    \item Gain of swapping $x$ and $y$: $gain(x,y) = (E_x - I_x) + (E_y - I_y) - 2\cdot w(e_{x,y})$
\end{itemize}

\begin{algorithm}
\caption{Kernighan-Lin}
\label{algo:kl}
\begin{algorithmic}[1]
    \State Partition randomly into $P_1$ and $P_2$
    \Comment Same size partitions
    \State Compute initial cutsize
    \Repeat
        \State Unlock all vertices
        \State Update costs $E_x$ and $I_x$
        \While {Unlocked cells}
            \ForAll{Unlocked vertices $x \in P_1$}
                \ForAll{Unlocked vertices $y \in P_2$}
                    \State Compute gain(x,y)
                \EndFor
            \EndFor
            \State Swap pair with highest gain
            \Comment Local search
            \State Lock those two vertices
        \EndWhile
        \State Accept $n$ first swaps leading to minimal cutsize
        \Comment Hill climbing
    \Until{Cutsize is not improved}
\end{algorithmic}
\end{algorithm}

\begin{Q}
Using the given template \texttt{q\_bonus\_kl\_template.py}, implement the KL partitioning algorithm.
Each line of the input file \texttt{graph.txt} describes an edge of the graph and is formatted as follows: \texttt{<edge weight> <vertex i> <vertex j>}
\reponse{
    See file \texttt{q\_bonus\_kl.py}
}
\end{Q}


\noindent
\rule{\textwidth}{0.4pt}
\footnotesize{Found an error? Let us know: \url{https://github.com/iridia-ulb/INFOH410/issues}}

\end{document} 
